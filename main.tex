% =================================================================================
% DOCUMENT CLASS & PACKAGES
% =================================================================================
\documentclass[11pt, a4paper]{article}

% --- Encoding and Font Packages ---
\usepackage[utf8]{inputenc} % Allows the use of UTF-8 characters

% --- Page Layout Packages ---
\usepackage{geometry}      % For setting page margins
\geometry{a4paper, margin=1in}

% --- Table and List Packages ---
\usepackage{array}         % Provides more advanced column formatting for tables
\usepackage{longtable}     % For tables that span multiple pages

% --- Graphics and Figures ---
\usepackage{graphicx}      % To include images

% --- Header and Footer Packages ---
\usepackage{fancyhdr}      % For custom headers and footers

% =================================================================================
% HEADER & FOOTER CONFIGURATION
% =================================================================================
\pagestyle{fancy}
\fancyhf{} % Clear all header and footer fields
\lhead{Software Requirements Specification for Campus Event Planner App}
\rfoot{Page \thepage}
\renewcommand{\headrulewidth}{0.4pt} % Adds a rule under the header
\renewcommand{\footrulewidth}{0pt}  % Removes the rule above the footer

% =================================================================================
% TITLE & AUTHOR INFORMATION
% =================================================================================
\title{
    \vspace{2cm} % Add some vertical space at the top
    \textbf{\huge Software Requirements Specification} \\
    \vspace{0.5cm}
    \large for \\
    \vspace{0.5cm}
    \textbf{\Huge Campus Event Planner App}
    \vspace{2cm}
}

\author{
    Prepared by: \\
    \textbf{Student 1 \& Student 2} \\
    \textit{School of Computing \& IT} \\
    \textit{Manipal University Jaipur}
}

\date{\textbf{August 12, 2025}}

% =================================================================================
% BEGIN DOCUMENT
% =================================================================================
\begin{document}

% --- Title Page ---
\begin{titlepage}
    \maketitle
    \vfill
    \begin{center}
        \textbf{Version 1.0 approved}
    \end{center}
\end{titlepage}

% --- Table of Contents ---
\pagenumbering{roman} % Start roman numbering for front matter
\tableofcontents
\newpage

% --- Revision History ---
\section*{Revision History}
\begin{tabular}{|m{4cm}|m{3cm}|m{6cm}|m{2cm}|}
    \hline
    \textbf{Name} & \textbf{Date} & \textbf{Reason For Changes} & \textbf{Version} \\
    \hline
    Student 1, Student 2 & 12/08/2025 & Initial Draft based on Sprint 0 Analysis & 1.0 \\
    \hline
\end{tabular}
\newpage

% =================================================================================
% MAIN CONTENT - Start Arabic Numbering
% =================================================================================
\pagenumbering{arabic}

% =================================================================================
% SECTION 1: INTRODUCTION
% =================================================================================
\section{Introduction}

\subsection{Purpose}
This Software Requirements Specification (SRS) document outlines the functional and non-functional requirements for the \textbf{Campus Event Planner App}, version 1.0. This product is a mobile-first web application designed to serve the students and staff of Manipal University Jaipur. Its purpose is to provide a centralized, efficient, and reliable platform for discovering, registering for, and managing attendance at campus events, thereby increasing student engagement.

\subsection{Document Conventions}
This document follows the IEEE 830-1998 standard for SRS documentation. Functional requirements are captured as User Stories and organized by features. The term "The system" refers to the Campus Event Planner App software.

\subsection{Intended Audience and Reading Suggestions}
This document is intended for the following audiences:
\begin{itemize}
    \item \textbf{Developers (Project Team):} To understand what needs to be built.
    \item \textbf{Project Manager (Course Instructor):} To oversee the project's progress.
    \item \textbf{Testers:} To create test cases based on the specified requirements.
\end{itemize}
It is recommended to read the Product Scope first, followed by the System Features and Nonfunctional Requirements.

\subsection{Product Scope}
The Campus Event Planner App aims to solve the problem of low event awareness and scattered information by creating a single, centralized hub for all official club, department, and university-wide events. The key goals are:
\begin{itemize}
    \item To increase student engagement and participation in campus events.
    \item To provide a single, reliable source of information for all campus activities.
    \item To simplify the event registration and management process for students.
    \item To provide event organizers with a simple way to promote their events and track attendance.
\end{itemize}
This project is being developed as the minor project for the CS-3230 Software Engineering Lab.

\subsection{References}
\begin{enumerate}
    \item \textit{CS-3230 Software Engineering Lab Course Handout, JAN-MAY 2022, Manipal University Jaipur}.
    \item \textit{Agile Alliance - User Story guide}.
    \item \textit{Atlassian - MoSCoW Prioritization guide}.
\end{enumerate}

% =================================================================================
% SECTION 2: OVERALL DESCRIPTION
% =================================================================================
\section{Overall Description}

\subsection{Product Perspective}
The Campus Event Planner is a new, self-contained product. It will operate as a standalone web application accessible to all members of the university community, with distinct functionalities for general users and event organizers.

\subsection{Product Functions}
The major functions of the system are captured in the Product Backlog:
\begin{itemize}
    \item User authentication and profile management.
    \item Event discovery (listing, searching, filtering).
    \item Event registration (RSVP) and personal schedule management.
    \item Event creation and management for organizers.
    \item Automated notifications for event reminders.
\end{itemize}

\subsection{User Classes and Characteristics}
\begin{longtable}{|p{4cm}|p{11cm}|}
    \hline
    \textbf{User Class} & \textbf{Characteristics} \\
    \hline
    \endfirsthead
    \hline
    \textbf{User Class} & \textbf{Characteristics} \\
    \hline
    \endhead
    \textbf{Student} & The primary user. Can browse and search for events, register for them, and view their personal event schedule. Expected to have basic web literacy. \\
    \hline
    \textbf{Event Organizer (Admin)} & A privileged user (e.g., club head, faculty coordinator). Can create, edit, and manage events, as well as view registration numbers for their events. \\
    \hline
\end{longtable}

\subsection{Operating Environment}
\begin{itemize}
    \item \textbf{Client-Side:} The system shall be a responsive web application accessible on modern browsers (Chrome, Firefox, Safari, Edge).
    \item \textbf{Server-Side:} The application will be developed using the MERN stack (MongoDB, Express.js, React, Node.js) and deployed on a cloud platform.
\end{itemize}

\subsection{Design and Implementation Constraints}
\begin{itemize}
    \item The system must be developed following the Agile/Scrum framework.
    \item All development must be version-controlled using Git and hosted on GitHub.
    \item The project must be completed within the 12-week semester timeline.
\end{itemize}

% =================================================================================
% SECTION 3: SYSTEM FEATURES (FUNCTIONAL REQUIREMENTS)
% =================================================================================
\section{System Features (Functional Requirements)}
The functional requirements are defined as a prioritized Product Backlog of User Stories.

\subsection{Must-Have Features (for First Release)}
\begin{description}
    \item[US-01:] As a student, I want to log in with my university credentials, so that I can access personalized features.
    \item[US-02:] As a user, I want to see a list of all upcoming events, so that I can find out what's happening on campus.
    \item[US-03:] As a user, I want to click on an event to see its details (date, time, location, description), so that I can decide if I want to attend.
    \item[US-04:] As an event organizer, I want to create and post a new event, so that students can see it.
\end{description}

\subsection{Should-Have Features}
\begin{description}
    \item[US-05:] As a student, I want to click an "RSVP" button, so that I can register for an event.
    \item[US-06:] As a student, I want to see a list of events I have registered for, so that I can keep track of my schedule.
    \item[US-07:] As a user, I want to search for events by name, so that I can find a specific event quickly.
    \item[US-08:] As an event organizer, I want to see how many students have registered for my event, so that I can plan accordingly.
\end{description}

\subsection{Could-Have Features}
\begin{description}
    \item[US-09:] As a user, I want to filter events by category (e.g., "Workshop", "Sports", "Music"), so that I can find events that match my interests.
    \item[US-10:] As a student, I want to receive a notification a day before an event I've registered for, so that I don't forget to attend.
    \item[US-11:] As a student, I want to have a simple profile with my name and email.
\end{description}

\subsection{Won't-Have Features (for this release)}
\begin{description}
    \item[US-12:] As a user, I want to view events on a calendar, so that I can see how they fit into my monthly schedule.
\end{description}

% =================================================================================
% SECTION 4: OTHER NONFUNCTIONAL REQUIREMENTS
% =================================================================================
\section{Other Nonfunctional Requirements}

\subsection{Performance Requirements}
\begin{itemize}
    \item The list of events must load in under 3 seconds on the campus Wi-Fi.
\end{itemize}

\subsection{Usability Requirements}
\begin{itemize}
    \item A new user should be able to find and view the details of an event in 3 clicks or less.
\end{itemize}

\subsection{Security Requirements}
\begin{itemize}
    \item All user passwords must be securely hashed and salted.
\end{itemize}
    
\subsection{Reliability Requirements}
\begin{itemize}
    \item The system should be available 99.5\% of the time.
\end{itemize}

% =================================================================================
% APPENDICES
% =================================================================================
\appendix
\newpage

\section{Acceptance Criteria for a Sample User Story}
This section provides a detailed example of the Acceptance Criteria for a high-priority user story, which makes it ready for development in Sprint 1.

\subsection*{User Story: US-02}
As a user, I want to see a list of all upcoming events, so that I can find out what's happening on campus.

\subsection*{Acceptance Criteria (Definition of Done)}
\begin{itemize}
    \item \textbf{Scenario 1: Viewing the event list} \\
    \textbf{Given} I am on the homepage, \\
    \textbf{When} the page loads, \\
    \textbf{Then} I should see a section titled "Upcoming Events".

    \item \textbf{Scenario 2: Event list displays correct information} \\
    \textbf{Given} I am viewing the "Upcoming Events" section, \\
    \textbf{When} I look at an event in the list, \\
    \textbf{Then} I should see the event's name, date, and time.

    \item \textbf{Scenario 3: Events are in chronological order} \\
    \textbf{Given} I am viewing the "Upcoming Events" section, \\
    \textbf{When} I look at the list, \\
    \textbf{Then} the events should be sorted with the soonest event at the top.

    \item \textbf{Scenario 4: No upcoming events} \\
    \textbf{Given} I am on the homepage, \\
    \textbf{And} there are no events scheduled in the future, \\
    \textbf{When} the page loads, \\
    \textbf{Then} I should see a message that says "No upcoming events at this time."
\end{itemize}

% =================================================================================
% END DOCUMENT
% =================================================================================
\end{document}
